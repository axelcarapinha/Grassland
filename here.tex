% !TeX program = xelatex -shell-escape

\documentclass {report}
\usepackage{amsmath} % for math symbols and more
\usepackage{graphicx} % for images
\usepackage{minted}   % syntax highlighting for code listings
\usepackage{titlepic} % picture in the title page
\usepackage{fontspec} % XelAtex and LuaTex font selection
\usepackage{enumitem}
%\usepackage[romanian]{babel}
\usepackage{listings}
\usepackage{tcolorbox}
\usepackage{clipboard}

\definecolor{codeblue}{RGB}{42,0,255}
\definecolor{codegreen}{RGB}{0,128,0}
\definecolor{codegray}{RGB}{128,128,128}
\definecolor{codepurple}{RGB}{160,32,240}
\definecolor{codebg}{RGB}{242,242,242}

\lstset{
  language=SQL,
  basicstyle=\ttfamily,
  keywordstyle=\color{codeblue},
  stringstyle=\color{codegreen},
  commentstyle=\color{codegray},
  numberstyle=\tiny\color{codepurple},
  breakatwhitespace=false,
  breaklines=true,
  captionpos=b,
  keepspaces=true,
  numbers=left,
  numbersep=5pt,
  showspaces=false,
  showstringspaces=false,
  showtabs=false,
  tabsize=2,
  backgroundcolor=\color{codebg},
}

\newclipboard{myclipboard}

% Custom environment with "COPY TO CLIPBOARD" button
\newtcolorbox{copiablecodebox}[1][]{%
  enhanced jigsaw, 
  breakable, 
  colback=codebg, 
  colframe=black, 
  boxrule=0.5pt, 
  sharp corners, 
  bottomrule=2pt, 
  toprule=0.5pt, 
  leftrule=0.5pt, 
  rightrule=0.5pt,
  borderline west={3pt}{0pt}{codeblue},
  arc=0pt,
  outer arc=0pt,
  title={#1},
  attach boxed title to top left={xshift=10pt, yshift=-3pt},
  minipage boxed title,
  boxed title style={%
    colback=codeblue,
    sharp corners,
    boxrule=0.5pt,
  },
  code upper*={\setbox0=\vbox\bgroup},
  code lower*={\egroup\begin{myclipboard}\unvbox0\end{myclipboard}},
}



\setmainfont[ItalicFont={timesi.ttf}, % support for italic
						BoldItalicFont={timesbd.ttf}]{times.ttf}
% \definecolor{light_blue}{RGB}{212,235,242}

% Edit table of contents links.
% \hypersetup{ 
%     colorlinks=true,  %set true if you want colored links
%     linktoc=all,      %set to all if you want both sections and subsections linked
%     linkcolor=black,  %choose some color if you want links to stand out
% }

\title {Relatório de Bases de Dados}

\titlepic{\includegraphics[width=\textwidth]{"logoUE.jpeg"}}
\author{
    Axel Amoroso Carapinha\\
    \texttt{55248}
    \and
    Leonel Figueira Caetano \\
  \texttt{00000}
}

\begin {document}
\date{}
\begin{figure}[t]
	\vspace {-2cm}
	\centering
	%\includegraphics[width=0.4\textwidth]{uni_logo.png}
\end{figure}
\maketitle

\renewcommand{\thesection}{\arabic{section}}
\makeatother

\tableofcontents 
\newpage

\section{Abstract}
	\hspace{1cm}Este trabalho teve como objetivo desenvolver uma base de dados 
    com base em relações já definidas e inserir dados na mesma.
    Além disso, deviam-se fazer queries de modo recolher informações 
    que nos poderiam interessar numa aplicação real do exemplo do enunciado.

\section{Resumo do projeto}
	\hspace{1cm}Devia-se criar uma base de dados para gerir a informação de uma  
    rede social de \textit{Fãs de Literatura}. Para se usar a mesma, 
    inseriram-se dados fictícios, e escreveram-se queries 
    que permitam recolher essa informação.

\section{Introdução}
	O objetivo deste relatório é mostrar as respostas do referido trabalho.

\section{Respostas}
\subsection{Questão 1}
\textit{Indique as chaves primárias, candidatas e estrangeiras de cada relação.}

\begin{enumerate}[label=-]

    \item \textbf{membro(Nome, IdMemb, Pais, Cidade, DataNasc)}
    \begin{enumerate}[label=\arabic*.]
        \item Chave primária
            \begin{itemize}
                \item \{IdMemb\}
            \end{itemize}
        \item Chave candidata
            \begin{itemize}
                \item \{IdMemb\}
            \end{itemize}
        \item Chaves estrangeiras
            \begin{itemize}
                \item NENHUMA
            \end{itemize}
    \end{enumerate}
    
    \item \textbf{amigo(IdMemb1, IdMemb2)}
    \begin{enumerate}[label=\arabic*.]
        \item Chaves primárias
            \begin{itemize}
                \item \{IdMemb1, IdMemb2\}
            \end{itemize}
        \item Chave candidata
            \begin{itemize}
                \item \{IdMemb1, IdMemb2\}
            \end{itemize}
        \item Chaves estrangeiras
            \begin{itemize}
                \item \{IdMemb1\}, que referencia a chave \{IdMemb\} na relação \textit{membro}
                \item \{IdMemb2\}, que referencia a chave \{IdMemb\} na relação \textit{membro}
            \end{itemize}
    \end{enumerate}
    
    \item \textbf{gosta(IdMemb, ISBN)}
    \begin{enumerate}[label=\arabic*.]
        \item Chave primária
            \begin{itemize}
                \item \{IdMemb, ISBN\}
            \end{itemize}
        \item Chave candidata
            \begin{itemize}
                \item \{IdMemb, ISBN\}
            \end{itemize}
        \item Chaves estrangeiras
            \begin{itemize}
                \item \{IdMemb\}, que referencia a chave \{IdMemb\} na relação \textit{membro}
                \item \{ISBN\}, que referencia a chave \{ISBN\} na relação \textit{livro}
            \end{itemize}
    \end{enumerate}
    
    \item \textbf{livro(ISBN, Titulo)}
    \begin{enumerate}[label=\arabic*.]
        \item Chave primária
            \begin{itemize}
                \item \{ISBN\}
            \end{itemize}
        \item Chave candidata
            \begin{itemize}
                \item \{ISBN\}
            \end{itemize}
        \item Chaves estrangeiras
            \begin{itemize}
                \item NENHUMA
            \end{itemize}
    \end{enumerate}
    
    \item \textbf{genero(ISBN, Genero)}
    \begin{enumerate}[label=\arabic*.]
        \item Chave primária
            \begin{itemize}
                \item \{ISBN, Genre\}
            \end{itemize}
        \item Chave candidata
            \begin{itemize}
                \item \{ISBN, Genre\}
            \end{itemize}
        \item Chave estrangeira
            \begin{itemize}
                \item \{ISBN\}, que referencia a chave \{ISBN\} na relação \textit{livro}
            \end{itemize}
    \end{enumerate}
    
    \item \textbf{autoria(ISBN, Coda)}
    \begin{enumerate}[label=\arabic*.]
        \item Chave primária
            \begin{itemize}
                \item \{ISBN, Coda\}
            \end{itemize}
        \item Chave candidata
            \begin{itemize}
                \item \{ISBN, Coda\}
            \end{itemize}
        \item Chaves estrangeiras
            \begin{itemize}
                \item \{ISBN\}, que referencia a chave \{ISBN\} na relação \textit{livro}
                \item \{Coda\}, que referencia a chave \{Coda\} na relação \textit{autor}
            \end{itemize}
    \end{enumerate}
    
    \item \textbf{autor(Coda, Nome, Pais)}
    \begin{enumerate}[label=\arabic*.]
        \item Chave primária
            \begin{itemize}
                \item \{CodA\}
            \end{itemize}
        \item Chave candidata
            \begin{itemize}
                \item \{CodA\}
            \end{itemize}
        \item Chaves estrangeiras
            \begin{itemize}
                \item NENHUMA
            \end{itemize}
    \end{enumerate}
\end{enumerate}


% \clearpage % Forces all figures above it in the .tex file to be printed before continuing with the text. This can leave large white spaces. More info in https://tex.stackexchange.com/questions/8625/force-figure-placement-in-text


\subsection{Questão 2}
\textbf{Ver ficheiro em anexo:} \textit{Tabelas+Dados.sql}.

\subsection{Questão 3}
\textbf{Ver ficheiro em anexo:} \textit{Tabelas+Dados.sql}.

\subsection{Questão 4}

\textbf{Views criadas para a resolução da questão:}

\begin{lstlisting}
    CREATE VIEW AgathaISBNs as (
      SELECT AutoriaISBN as ISBN
      FROM Autoria JOIN Autor ON AutoriaCoda = Autor.Coda
      WHERE Autor.Nome = 'Agatha Christie'
    );

    -- Usada em b), d), g), h)
    CREATE VIEW GostaAgatha AS (
      SELECT DISTINCT GostaIdMemb as IdMemb
      FROM Gosta JOIN AgathaISBNs ON GostaISBN = AgathaISBNs.ISBN
    );
\end{lstlisting}
\[
\text{AgathaISBNs} \leftarrow \pi_{\text{ISBN}}\left(\rho_{\text{ISBN/AutoriaISBN}}\left(\sigma_\text{Nome = 'Agatha Christie'}}(\text{Autor} \bowtie_{\text{AutoriaCoda = Coda}} \text{Autoria})\right)\right)
\]
\[
\text{GostaAgatha} \leftarrow \pi_{\text{GostaIdMemb}}\left(\delta_{\text{IdMemb, ISBN}}(\text{Gosta} \bowtie_{\text{GostaISBN = ISBN}} \text{AgathaISBNs})\right)
\]
\vspace{0.5cm}

\begin{lstlisting}   
    -- Usada em e), f), i), q) 
    CREATE VIEW AmigosOleitor AS (
      SELECT IdAmigo1 AS IdMemb
      FROM Amigo 
      WHERE IdAmigo2 = 'oleitor'
       
      UNION
       
      SELECT IdAmigo2 AS IdMemb
      FROM Amigo 
      WHERE IdAmigo1 = 'oleitor'
    );
\end{lstlisting}
\[
\text{AmigosOleitor} \leftarrow \pi_{\text{IdAmigo1}}(\sigma_{\text{IdAmigo2 = 'oleitor'}}(\text{Amigo})) \cup \pi_{\text{IdAmigo2}}(\sigma_{\text{IdAmigo1 = 'oleitor'}}(\text{Amigo}))
\]
\vspace{0.5cm}


    
\begin{lstlisting}    
    CREATE VIEW FranciscoISBNs as (
      SELECT AutoriaISBN as ISBN
      FROM Autoria JOIN Autor ON AutoriaCoda = Autor.Coda
      WHERE Autor.Nome = 'Francisco José Viegas'
    );

    -- Usada em g), h)
    CREATE VIEW GostaFrancisco AS (
      SELECT DISTINCT GostaIdMemb as IdMemb
      FROM Gosta JOIN FranciscoISBNs ON GostaISBN = FranciscoISBNs.ISBN
    );
\end{lstlisting}
\[
\text{FranciscoISBNs} \leftarrow \pi_{\text{AutoriaISBN}}\left(\sigma_{\substack{ \\ \text{Nome = 'Francisco José Viegas'}}}(\text{Autoria} \bowtie_{\text{AutoriaCoda = Autor.Coda}} \text{Autor})\right)
\]
\[
\text{GostaFrancisco} \leftarrow \pi_{\text{GostaIdMemb}}(\delta_{\text{IdMemb, ISBN}}(\text{Gosta} \bowtie_{\text{GostaISBN = ISBN}} \text{FranciscoISBNs}))
\]
\vspace{0.5cm}


\begin{lstlisting}    
    -- Usada em l), m), n)
    CREATE VIEW LivrosStats AS (
      SELECT l.ISBN as ISBN,
             l.Titulo,
             COUNT(DISTINCT Genero.Genero)     as numGeneros,
             COUNT(DISTINCT Gosta.GostaIdMemb) as numGostos
      FROM Livro AS l, Genero, Gosta
      WHERE l.ISBN = GeneroISBN AND
            l.ISBN = GostaISBN
      GROUP BY ISBN
      ORDER BY ISBN
    );
\end{lstlisting}
\textit{Álgebra relacional no fim do documento.}


\vspace{0.5cm}

\begin{lstlisting}    
    CREATE VIEW AmigosSimetrico AS (
        SELECT IdAmigo1 as IdMemb, IdAmigo2 as Idamigo
        FROM Amigo 
        
        UNION
        
        SELECT idamigo2 as IdMemb, IdAmigo1 as idAmigo
        FROM Amigo 
        );
        
    -- Usada em o), p)
    CREATE VIEW numAmigos AS (
      SELECT idmemb, COUNT(Idamigo) as  n
      FROM AmigosSimetrico
      GROUP BY idmemb
      ORDER BY n DESC
    );

    COMMIT;
\end{lstlisting}
   \[
   \text{AmigosSimetrico} \leftarrow \pi_{\text{IdAmigo1, IdAmigo2}}(\text{Amigo}) \cup \pi_{\text{idamigo2, IdAmigo1}}(\text{Amigo})
   \]
   \[
   \text{numAmigos} \leftarrow \pi_{\text{idmemb, n}}\left(\gamma_{\text{idmemb, COUNT(Idamigo) as n}}(\text{AmigosSimetrico})\right)
   \]
\vspace{0.5cm}

\textit{Indique a expressão em SQL e em Álgebra Relacional para responder às seguintes perguntas:}

\subsection{a)}
\textit{Qual é o nome dos autores de obras do género drama? (se não inseriu
o género drama, insira-o e insira também este género em pelo menos
uma obra)}
\[
\pi_{\text{Autor.Nome}}\left(\sigma_{\text{Genero} = \text{'Drama'} \land \text{GeneroISBN} = \text{AutoriaISBN} \land \text{Coda} = \text{AutoriaCoda}}(\text{Autor} X \text{Genero} X \text{Autoria})\right)
\]

\subsection{b)}
\textit{Qual o nome dos membros que gostam de livros da Agatha Christie?
(se não inseriu a autora insira e insira também pelo menos um gosto
de um membro numa obra da autora}
\[
\pi_{\text{Membro.nome}}\left(\sigma_{\text{Membro.IdMemb} \in \pi_{\text{IdMemb}}(\text{GostaAgatha})}(\text{Membro})\right)
\]

\subsection{c)}
\textit{Qual o nome dos membros que gostam de um livro de um autor que nasceu no seu país?
(se não inseriu o autor, insira e adicione pelo menos um gosto de um membro numa obra do autor)}
\[ 
\pi_{\text{Membro.nome}}\left(\sigma_{\substack{\text{Membro.IdMemb} = \text{GostaIdMemb} \\ \land \\ \text{Autoria.autoriaisbn} = \text{GostaISBN} \\ \land \\ \text{AutoriaCoda} = \text{Autor.Coda} \\ \land \\ \text{Membro.Pais} = \text{Autor.Pais}}}(\text{Membro} X \text{Gosta} X \text{Autoria} X \text{Autor})\right)
\]

\subsection{d)}
\textit{Quais os membros que NÃO gostam de algum livro da Agatha Christie?}
\[
\pi_{\text{Membro.nome}}\left(\sigma_{\text{Membro.IdMemb} \notin \pi_{\text{IdMemb}}(\text{GostaAgatha})}(\text{Membro})\right)
\]

\subsection{e)}
\textit{Quais os membros que NÃO são amigos do membro que com o idMemb\texttt{oleitor}?}
\[
\pi_{\text{Membro.nome}}\left(\sigma_{\text{Membro.IdMemb} \notin \pi_{\text{IdMemb}}(\text{AmigosOleitor})}(\text{Membro})\right)
\]

\subsection{f)}
\textit{Qual o nome dos amigos do \texttt{oleitor} que são mais jovens que ele?}
\textit{Álgebra relacional no fim do documento.}

\subsection{g)}
\textit{Qual o nome dos membros que gostam de livros da Agatha Christie e do Francisco José Viegas?}
\[
\pi_{\text{Membro.nome}}\left((\sigma_{\text{GostaAgatha.IdMemb = Membro.IdMemb}}(\text{Membro} \bowtie \text{GostaAgatha}) \cap \sigma_{\text{GostaFrancisco.IdMemb = Membro.IdMemb}}(\text{Membro} \bowtie \text{GostaFrancisco}))\right) \bowtie_{\text{Membro.nome}} \text{Membro}
\]
\textit{Álgebra relacional no fim do documento.}
%
\subsection{h)}
\textit{Qual o nome dos membros que gostam de livros da Agatha Christie ou do Francisco José Viegas?}
\[
\pi_{\text{Membro.nome}}\left(\pi_{\text{Membro.nome}}(\sigma_{\text{Membro.IdMemb} = \text{GostaAgatha.IdMemb}}(\text{Membro} \bowtie \text{GostaAgatha})) \cup \pi_{\text{Membro.nome}}(\sigma_{\text{Membro.IdMemb} = \text{GostaFrancisco.IdMemb}}(\text{Membro} \bowtie \text{GostaFrancisco}))\right)
\]
\textit{Álgebra relacional no fim do documento.}

\subsection{i)}
\textit{Quantos amigos tem o membro \texttt{oleitor}?}
\[
\text{COUNT}(\pi_{\text{IdMemb}}(\text{AmigosOleitor}))
\]

\subsection{j)}
\textit{Qual é o membro que tem mais amigos?}
\[
\pi_{\text{Membro.Nome}}\left(\text{Membro} \bowtie \text{MembroMaisAmigos}\right)
\]

\subsection{k)}
\textit{Qual o nome dos membros que são amigos do membro que gosta de mais livros?}
\[
\pi_{\text{Membro.Nome}}\left(\pi_{\text{Membro.Nome}}(\sigma_{\substack{\text{Membro.IdMemb} = \text{Amigo.IdAmigo1} \\ \text{e} \\ \text{Amigo.IdAmigo2} = \text{MaiorFaLivros.IdMemb}}}\text{Membro} \bowtie \text{Amigo} \bowtie \text{MaiorFaLivros} \right) \cup \pi_{\text{Membro.Nome}}\left(\pi_{\text{Membro.Nome}}(\sigma_{\substack{\text{Membro.IdMemb} = \text{Amigo.IdAmigo2} \\ \text{e} \\ \text{Amigo.IdAmigo1} = \text{MaiorFaLivros.IdMemb}}}\text{Membro} \bowtie \text{Amigo} \bowtie \text{MaiorFaLivros} \right)
\]
%

\subsection{l)}
\textit{Para cada livro, indique o número de géneros.}
\[
\pi_{\text{Titulo, numGeneros}}\left(\text{LivrosStats} \right)
\]

\subsection{m)}
\textit{Para cada livro, indique o número de géneros e o número de gostos.}
\[
\pi_{\text{Titulo, numGeneros, numGostos}}\left(\text{LivrosStats} \right)
\]

\subsection{n)}
\textit{Para cada autor, indique o número de livros, o número de géneros e o número de gostos.}
\[
\pi_{\text{Autor.Nome, COUNT(DISTINCT AutoriaISBN) as NumLivros, COUNT(DISTINCT GenerosAutores.Genero) AS NumGeneros, SUM(DISTINCT numGostos) as NumGostos}}\left(\text{Autoria} \bowtie \text{Autor} \bowtie \text{GenerosAutores} \bowtie \text{LivrosStats} \right)
\]
\textit{Álgebra relacional no fim do documento.}


\subsection{o)}
\textit{Para cada membro, indique o número de amigos e o número de livros de que gosta.}
\[
\pi_{\text{Membro.Nome, NumAmigos.n as NumAmigos, COUNT(GostaISBN) as NumLivrosGosta}}\left(\text{Membro} \bowtie \text{NumAmigos} \bowtie \text{Gosta} \right)
\]
\textit{Álgebra relacional no fim do documento.}

\subsection{p)}
\textit{Qual o nome dos membros que são amigos de todos os membros?}
\[
\text{AmigosTodosMembros} \leftarrow \pi_{\text{IdMemb}}\left(\sigma_{\text{n = NumMembros}}\left(\text{NumAmigos} \bowtie_{\text{n = NumMembros}} \left(\pi_{\text{COUNT(Membro.IdMemb) - 1 as NumMembros}}(\text{Membro})\right)\right)\right)
\]
\textit{Álgebra relacional no fim do documento.}


\[
\pi_{\text{Membro.nome}}\left(\text{Membro} \bowtie \text{AmigosTodosMembros} \right)
\]


\subsection{q)}
\textit{Quais os títulos dos livros de que todos os amigos do leitor gostam?}
\[
\pi_{\text{Titulo}}(Livro) \, \div \, \pi_{\text{IdMemb}}(\text{AmigosOleitor} \, \ominus \, \pi_{\text{GostaIdMemb, GostaIsbn}}(\text{Gosta} \, \bowtie_{\text{GostaIsbn = l.Isbn}} \, \pi_{\text{Isbn}}(l)))
\]
\textit{Álgebra relacional no fim do documento.}
%


\section{Conclusão}
\hspace{1cm}Neste trabalho conseguimos concluir todos os objetivos com sucesso,
desde a criação da base de dados às queries para se recolher informação.
Consideramos como principal ponto extra o facto de nos termos preocupado
em fazer de vários modos diferentes, aprendendo as formas mais eficazes possíveis.
	

\end {document}
